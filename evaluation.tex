\subsection{Evaluation}
\label{sec:appevaluation}
These are the examples used to evaluate HASL/1 and HASL/2. They are chosen to cover the requirements mentioned in the introduction. These examples are also available in the online versions of HASL.

\autoref{table:haslevaluation} lists evaluation of HASL/1 and HASL/2 per example. For each example we noted the number of interpretations provided by HASL/1, and the number we judged to be correct. Any remarks on what went right or wrong are noted as well.

\subsubsection{Tort}
\begin{exe}
\ex\label{e126} A person has the duty to repair another person's damages if the other person has suffered damages by someone else's act, the act committed is tortious, the act can be imputed to the person and the damages are caused by the act.
\ex\label{e127} A person has the duty to repair another person's damages if the other person has suffered damages by someone else's act, if the act committed is tortious, if the act can be imputed to the person or if the damages are caused by the act.
\ex\label{e128} A person has the duty to repair another person's damages if the other person has suffered damages by someone else's act, if the act committed is tortious or if the act can be imputed to the person and the damages are caused by the act.
\ex\label{e129} A person has the duty to repair another person's damages if the other person has suffered damages by someone else's act, if the act committed is tortious, if the act commited is real and the person is real or if the act can be imputed to the person and the damages are caused by the act.
\ex\label{e130} A person has the duty to repair another person's damages if the other person has suffered damages by someone else's act, if the act committed is tortious or if the act can be imputed to the person and the damages are caused by the act. The act committed is tortious if there was a violation of someone else’s right, if an act or omission is in violation of a duty imposed by law or if an act or omission is in violation of what according to unwritten law has to be regarded as proper social conduct.
\ex\label{e131} A person has the duty to repair another person's damages if the other person has suffered damages by someone else's act, the act committed is tortious, the act can be imputed to the person and the damages are caused by the act.\\ The act committed is tortious if the act is a violation of someone’s right unless there was a ground of justification, if the act is a violation of a statutory duty unless there was a ground of justification or if the act is a violation of unwritten law against proper social conduct unless there was a ground of justification.\\ The act can be imputed to the person if the act results from the person's fault, if the act results from a cause for which he is accountable by virtue of law or if the act results from a cause for which he is accountable by virtue of generally accepted principles.
\end{exe}

\clearpage\input{appendix/evaluationresults}