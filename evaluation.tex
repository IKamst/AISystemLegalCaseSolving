\subsection{Evaluation}
\label{sec:appevaluation}
These are the examples used to evaluate HASL/1 and HASL/2. They are chosen to cover the requirements mentioned in the introduction. These examples are also available in the online versions of HASL.

\autoref{table:haslevaluation} lists evaluation of HASL/1 and HASL/2 per example. For each example we noted the number of interpretations provided by HASL/1, and the number we judged to be correct. Any remarks on what went right or wrong are noted as well.

\subsubsection{Conceptual laws}
\begin{exe}
\ex\label{1.1} A if B.
\ex\label{1.2} A if B and C.
\ex\label{1.3} A if B or if C.
\ex\label{1.4} A if B. B if C.
\end{exe}

\subsubsection{Tort}
\begin{exe}
\ex\label{2.1} A person has the duty to repair another person's damages if the other person has suffered damages by someone else's act, the act committed is tortious, the act can be imputed to the person and the damages are caused by the act. The act committed is tortious if the act is a violation of someone’s right, if the act is a violation of a statutory duty or if the act is a violation of unwritten law against proper social conduct. The act can be imputed to the person if the act results from the person's fault, if the act results from a cause for which he is accountable by virtue of law or if the act results from a cause for which he is accountable by virtue of generally accepted principles.
\end{exe}

\clearpage\input{appendix/evaluationresults}